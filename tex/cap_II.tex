\chapter{Preliminares}
\label{cap:cap1}

Neste capítulo será realizada uma abordagem a conceitos que não são temas principais deste trabalho, porém, que de alguma forma auxiliam na compreensão e implementação da arquitetura proposta.

Aqui é explorado o conceito de \textit{Big Data} que motivou e originou o problema, ao qual propomos uma solução utilizando o FS-Join. Ainda será encontrada uma abordagem do \textit{framework} \textit{Spark Apache}, realizada uma breve comparação entre o \textit{MapReduce} e ele, buscando explicar os conceitos básicos por trás do \textit{Spark} e realizar uma \textit{overview} sobre as funções utilizadas na implementação. E ainda nesta mesma seção, poderá ser encontrada a motivação para se utilizar a linguagem de programação \textit{Scala}.

\section{\textit{Big Data}}

Não há um consenso entre os autores da área sobre a definição do termo \textit{Big Data} e há um motivo pra isso. Segundo \cite{doi:10.1108/LR-06-2015-0061} o motivo para a não existência uma definição única se deve à rápida evolução da literatura de \textit{Big Data} que impediu o desenvolvimento de uma definição universal e formalmente aceita para o termo.

\textit{Big Data} nasceu com conceito ligado à característica dos dados, que são os \textit{3Vs} -- Volume, Velocidade e Variedade. Passou por conceitos ligados à tecnologia e métodos analíticos necessários para ser fazer o devido uso destes dados. E por conceitos relacionados ao valor que estes dados podem gerar quando há transformação dos dados obtidos em informações, e das informações em valor para o negócio do qual os dados foram obtidos.

Tomaremos neste trabalho a definição de \textit{Big Data} sugerida por \cite{doi:10.1108/LR-06-2015-0061}:  "O \textit{Big Data} é o ativo de informação caracterizado por um volume, velocidade e variedade tão elevados que requer tecnologia específica e métodos analíticos para sua transformação em valor."

Consideramos que volume não está relacionado somente a quantidade de dados gerados ou obtidos através das atividades de \textit{Big Data}, mas também, ao volume de operações necessárias para a higienização dos dados para o devido armazenamento.

\section{Conceitos de operações por similaridade}

Esta seção é destinada à elucidação de conceitos relacionados à operações por similaridade, utilizadas neste trabalho.

\subsection{\textit{Tokenização}}

O processo de \textit{tokenização} é fundamental para aplicações em tarefas de processamento de linguagem natural, como descrito por Bird et al. (2009) \cite{Bird:2009:NLP}, e amplamente utilizado em analises textuais para conversão destes dados em conjuntos de \textit{strings} (do inglês, "cadeia") de caracteres. Estes conjuntos serão compostos por \textit{substrings} (sub-cadeias de caracteres) indivisíveis, ou seja, estes conjuntos serão compostos por partes do texto original na ordem em que aparecem, cada elemento deste conjunto recebe o nome de \textit{token}.  Por este motivo é dado o nome abrasileirado de \textit{tokenização} (no inglês, chamado de \textit{tokenization}).

Para exemplificar o processo de \textit{tokenização}, tomemos cada palavra (no sentido textual) em uma \textit{string} como sendo um \textit{token}, neste contexto, seja "As maquinas irão dominar o mundo." a \textit{string} utilizada como entrada, teremos como saída \{"As","maquinas","irão","dominar","o","mundo."\}. Entretanto aqui o objetivo é utilizar a técnica de \textit{tokenização} para encontrar duplicatas ao comparar duas \textit{strings}, e realizar a \textit{tokenização} em nível tão macro, em que cada \textit{token} corresponda à uma palavra, não será eficiente para verificar caso hajam erros de digitação em apenas um caractere do \textit{token}.

Por este motivo a técnica de \textit{tokenização} utilizada neste trabalho é uma adaptação da técnica de \textit{q-gram} posicional, proposta por Gravano et al. (2001) \cite{Gravano:2001:ASJ:645927.672200}. O processo de criação de \textit{tokens} baseado em \textit{q-grams}, consiste em criar \textit{substrings} deslizando uma janela de comprimento $q$ definida sobre os caracteres das \textit{strings} fornecidas. Com a finalidade de que os caracteres da \textit{string} apareçam exatamente a mesma quantidade de vezes nos \textit{tokens}, são adicionados $q-1$ os novos caracteres  especiais no incio e no fim das \textit{strings}, antes de realizar o processo de \textit{tokenização}. A técnica de \textit{q-gram} posicional, proposta por Gravano et al. (2001) \cite{Gravano:2001:ASJ:645927.672200}, adiciona a cada \textit{token} sua posição na \textit{string} original, neste trabalho sinalizaremos os \textit{tokens} com a ordem de aparição, para o caso de \textit{tokens} iguais porém em posições diferentes, veja mais no exemplo que segue .

Tomemos como \textit{string} de entrada a palavra "amargar" para exemplificar como é realizado o processo neste trabalho, iremos aplicar a \textit{tokenização} fazendo $q=2$, como caractere especial será utilizado "\#", (1) iremos adicionar $q-1$ caracteres especiais no inicio e fim da \textit{string}, portanto, serão adicionados $1$($2-1=1$) caracteres e teremos como saída "\#AMARGAR\#"; (2) o segundo passo é realizar a \textit{tokenização} com temos uma $2-gram$, então, a janela terá comprimento $2$, e portanto como resultado teremos o conjunto \{"\#A1","AM1","MA1","AR1","RG1","GA1","AR2","R\#1"\}. Como há duas ocorrências para o \textit{token} "AR", então, a primeira ocorrência recebe $1$, e a segunda ocorrência recebe $2$, com a finalidade de verificar a posição à título de comparação de duas strings.

Este processo de \textit{tokenização} é fundamental para o processo de junção por similaridade e comparação entre duas \textit{strings}, portanto, fundamental para o bom entendimento e execução da proposta deste trabalho.

\subsection{Junção por similaridade}

No contexto deste trabalho, é necessário entender junção por similaridade como sendo uma operação de comparação entre duas \textit{strings} obtidas à partir de um conjunto de dados de entrada textuais, compara-las com a fim de calcular suas pontuações de similaridade e verificar se suas pontuações não são menores que um determinado valor limite, aqui chamaremos este valor limite de \textit{threshold}, e o representaremos pela letra grega $\theta$(\textit{theta}). Ou seja, dados duas strings $S$ e $V$, aplicadas à uma determinada função de similaridade $sim(S,V)$, dizemos que são similares, caso o resultado dessa função seja maior ou igual à um determinado \textit{threshold}, portanto, são similares caso $sim(S,V) \geq\theta$.

Existem três funções conhecidas para o cálculo da similaridade entre conjuntos, são elas \textit{Jaccard} proposto por Monge et al. (1996)\cite{Monge:1996}[14], \textit{Dice} proposto por Salto et al. 1986\cite{Salton:1986} e Cosseno proposto por Winkler (1999) \cite{Winkler:1999}, cujas definições estão na Tabela \ref{tabelaFuncoesSimilaridade}.

\begin{table}[H]
\centering
 \caption{Funções de Similaridade Baseada em Conjuntos, em que $T$ representa o conjunto de \textit{tokens} de cada \textit{string}}
  \label{tabelaFuncoesSimilaridade}
    \begin{tabular}{|l|c|} 
    \hline Função & Definição \\
    \hline $sim_{jaccard}(S,V)$ & $\dfrac{|T_S \cap T_V|}{|T_S \cup T_V|}$\\ 
    \hline $sim_{dice}(S,V)$ & $\dfrac{2*|T_S \cap T_V|}{|T_S| + |T_V|}$ \\ 
    \hline $sim_{cosseno}(S,V)$ & $\dfrac{|T_S \cap T_V|}{|T_S| * |T_V|}$ \\
    \hline
    \end{tabular}
\end{table}

Neste trabalho é considerado como função de cálculo de similaridade padrão a função \textit{Jaccard}, ou seja, ao referirmos $sim(S,V) \geq\theta$, estamos dizendo que, $sim_{jaccard}(S,V) \geq\theta$.

\section{O \textit{framework Spark Apache}}

Foram consideradas duas ferramentas \textit{open source} para serem utilizadas na elaboração deste trabalho. O \textit{Hadoop} com seu famigerado \textit{MapReduce}, e o Spark com suas \textit{RDDs}, ambas da comunidade Apache. Entretanto, foi selecionado o \textit{framework} \textit{Spark}.

O \textit{framework}\textit{ Spark Apache} possui um arcabouço de ferramentas, e neste trabalho foram utilizadas apenas apenas  as que fornecem a abstração de processamentos paralelos de grandes volumes dados e que podem substituir as primitivas do \textit{MapReduce}.

\textit{MapReduce} e \textit{Spark} são duas estruturas de computação em \textit{cluster}\footnote{\textit{Cluster} é um termo amplamente usado que significa computadores independentes combinados em um sistema unificado por meio de \textit{software} e rede. No nível mais fundamental, quando dois ou mais computadores são usados juntos para resolver um problema, ele é considerado um \textit{cluster}. Os \textit{clusters} geralmente são usados para alta disponibilidade (\textit{HA - High Availability}) para maior confiabilidade ou alta performance computacional (\textit{HPC -  High Performance Computing}) para fornecer maior poder computacional do que um único computador pode fornecer.\cite{1401797-Cluster}} livre, populares para análise de dados em larga escala. Ambas ferramentas ocultam a complexidade na realização de processamentos paralelos e da tolerância a falhas, expondo aos usuários uma \textit{API} (\textit{Application Programming Interface} - em tradução livre "Interface de Programação de Aplicativos") de programação simples. 

Em experimentos realizados em \cite{MpReVsSpark}, o processamento de dados foi explorado de várias formas, com a finalidade de comparar as duas ferramentas. No qual comprovaram que o \textit{MapReduce} possui maior eficácia que o \textit{Spark} em operações de ordenação do tipo \textit{Sort}, e \cite{MpReVsSpark} atribuem isso ao fato de que o \textit{Spark} utiliza clusterização dos dados processados, trabalhando em nós de processamentos em memória, enquanto o \textit{MapReduce} utiliza-se muito de processamentos utilizando o disco. Se faz necessário observar que essa diferença pode ser reduzida dependendo do tipo de disco que será utilizado.

\subsection{\textit{Dataset}}

Antes de descrever do que se trata um \textit{RDD}, é conveniente entender o \textit{dataset} segundo a documentação do \textit{Spark}\cite{SparkPage}. O \textit{dataset} é uma coleção de objetos de domínio específico, fortemente tipada  e podem ser paralelizados utilizando operações funcionais ou relacionais. Cada \textit{dataset} possui uma visão não tipada chamada \textit{DataFrame}, que é o conjunto de dados de uma \textit{Row}. A \textit{Row} representa um vetor com o resultado de um operador contendo um ou mais valores, aceitando valores de tipos diferentes, sejam eles do tipo chave/valor, ou vários valores, por exemplo, $\textit{Row}1(valor1,valor2,valor3)$, onde os valores podem ser acessados pelo índice da posição, aonde o índice 0 corresponde ao valor1 (${Row}1[0]=valor1$), o índice 1 corresponde ao valor2  (${Row}1[1]=valor2$) e assim por diante.

Há operações disponíveis nos \textit{Datasets}, estão dividas entre ações e transformações. Transformações agem sobre o \textit{Dataset} e geram um novo \textit{Dataset}, enquanto ações acionam operações computacionais sobre os \textit{datasets} e retornam os resultados. São exemplos de transformações:  mapear (\textit{map}), filtrar (\textit{filter}), selecionar (\textit{select}) e agregar (\textit{groupBy}). E são exemplos de ação: contadores (\textit{count}), mostrar os dados (\textit{show}), ou gravando dados em arquivos de sistemas.

Este conceito de \textit{dataset} auxilia no melhor entendimento do \textit{RDD} que será discutido na seção à seguir.

\subsection{\textit{RDD} - \textit{Resilient Distributed Dataset}}

Entenda \textit{RDD (Resilient Distributed Dataset)} como um conjunto de dados resilientes e distribuídos, que nada mais é do que uma abstração de memória distribuída, fornecendo aos programadores a possibilidade de executar cálculos em memória de grandes \textit{clusters} de maneira tolerante a falhas.  

A origem dos \textit{RDDs} foi motivada por aplicações que lidam de forma ineficiente com algorítimos interativos e ferramentas interativas para mineração de dados. E que segundo \cite{Zaharia:2012:RDD} , em ambos os casos, o desempenho de manter dados em memória pode melhorar o desempenho em uma ordem de grandeza. 

A tolerância a falhas é eficiente, pois, os \textit{RDDs} fornecem memória compartilhada de forma restrita, ou seja, com base em transformações \textit{coarse-grained} (de alta granularidade), em vez de atualizações \textit{fine-grained} (de baixa granularidade) para o estado compartilhado dos \textit{datasets}. Para deixar a comparação visualmente mais simples, tomemos como exemplo um \textit{dataset} com um milhão de linhas, para este caso uma operação \textit{fine-grained} seria aplicada no menor conjunto possível, no caso uma linha, enquanto que, em uma operação \textit{coarse-grained} seria aplicada sobre o \textit{dataset} como um todo.

No entanto, \cite{Zaharia:2012:RDD} mostra que os \textit{RDDs} são expressivos o suficiente para capturar uma ampla classe de cálculos, incluindo modelos de programação especializados para trabalhos iterativos.

\subsection{Linguagem de Programação \textit{Scala}}

Cabe aqui, descrever o que motivou a escolha da linguagem \textit{Scala} para a realização da implementação do \textit{FS-Join}, mesmo a linguagem não sendo tão popular.

A linguagem \textit{Scala} foi criada em 2001-2004 na \textit{EPFL} (\textit{École Polytechnique Fédérale de Lausanne})  em seu laboratório de métodos de programação, com a finalidade de aprimorar o suporte à linguagens de componentes de \textit{software}. Esses estudos foram orientados principalmente para duas ideias. Primeiro, a linguagem de aplicação do componente deveria ser escalável, ou seja, deveria ser possível usar os mesmos conceitos tanto para escrever pequenas parte de um aplicativo, quanto para escrever grandes partes do mesmo. Os desenvolvedores da linguagem estavam focados implementar de maneira eficiente a abstração, composição, e mecanismos de decomposição, em vez de introduzir um grande número de tipos primitivos, úteis somente em altos níveis de escalabilidade. Em segundo lugar, é bastante natural criar uma linguagem que unifique e generalize orientação a objeto com programação funcional. Uma das principais inovações técnicas da linguagem \textit{Scala} é o conceito que faz a junção entre esses dois paradigmas \cite{Glybovets2010}.

A linguagem \textit{Scala} não fornece primitivas de comunicação e
sincronização para programação paralela. O núcleo da linguagem é construído de forma a simplificar a criação de bibliotecas que suportam vários modelos de paralelismo construídos sobre o modelo atual da linguagem principal \cite{Glybovets2010}.

Existe certa semelhança entre os programas escritos em \textit{Java} e os escritos em \textit{Scala}. Os programas escritos em \textit{Java} podem interagir livremente com os códigos escritos em \textit{Scala} e vice-versa. Programas escritos em \textit{Scala} podem substituir partes críticas de código escritos em \textit{Java}, que exigem alto nível de processamento paralelo ou escrita funcional, e ainda ambos permanecerem dentro do mesmo projeto, pois a linguagem \textit{Scala} roda sobre o \textit{JVM} (\textit{Java Virtual Machine}), da \textit{Oracle}. Havia uma compatibilidade do Scala com o .\textit{NET} \textit{Framework}, da \textit{Microsoft}, e que porém foi descontinuada nas versões mais recentes da linguagem.

Portanto, por se tratar de uma linguagem que fornece boas ferramentas nativas para processamentos paralelos, e por ainda ser nativamente compatível com o \textit{Framework} \textit{Spark Apache}, além de integrar com projetos \textit{Java} pré-existentes, a mesma foi selecionada para integrar o arcabouço de ferramentas utilizadas neste trabalho.

Mais informações sobre a linguagem podem ser obtidas na documentação no site próprio da mesma \cite{ScalaPage}.

\subsection{O \textit{MapReduce}}

Esta seção se faz importante pelo fato que o trabalho \cite{Rong:2017:FS-Join}, no qual se baseia este trabalho, utilizou a arquitetura do \textit{MapReduce} para realizar a implementação do \textit{framework} proposto e este trabalho propõe faze-lo utilizando o \textit{framework} \textit{Spark Apache}. Na próxima seção será descrito o processo análogo para o \textit{Spark Apache}.

Segundo \cite{Rong:2017:FS-Join} o \textit{MapReduce} foi proposto para facilitar o processamento de conjuntos de dados de larga escala em \textit{clusters} não compartilhados, utilizando computadores comerciais não especializados. Por sua escalabilidade e tolerância a falhas, ela se tornou a plataforma de fato para o processamento de \textit{Big Data}.

O \textit{MapReduce} consiste em duas primitivas, \textit{Map} (mapear) e \textit{Reduce} (reduzir). Na estrutura do \textit{MapReduce}, todos os registros são representados como pares chave/valor (\textit{key/value}). O conjunto de dados de entrada geralmente é armazenado em um sistema de arquivos distribuído (\textit{DFS - Distributed File System}), em vários nós de execução da carga de trabalho. Depois que a carga de trabalho  é enviada para o MapReduce, os registros são fornecidos aos nós do mapeador em partes. Cada registro será analisado no par de chaves/valores \textit{<key1, value1>} e é aplicada a função de \textit{Map} que cria uma nova lista com pares de chave/valor \textit{lista(<key2, value2>).} Os pares de chave/valor com a mesma chave serão enviados para o mesmo nó de redução durante a fase de reprodução aleatória e serão agrupados por sua chave na forma de lista de valores \textit{<key2, lista(value2)>}. O redutor receberá a lista de valores \textit{<key2, lista(value2)>} e aplicará a função \textit{Reduce} a eles. Por fim, o redutor produzirá uma lista de pares de chave/valor \textit{lista(<key3, value3>)} e os gravará no sistema de arquivos distribuído. O processo acima pode ser formalizado como abaixo.

\begin{itemize}
\item  Map: $\textit{<key1,value1>} \rightarrow \textit{lista(<key2, value2>)}$;
\item Reduce: $\textit{<key2, lista(value2)>} \rightarrow \textit{lista(<key3, value3>)}$;
\end{itemize}

Nesta seção será discutida as funções do \textit{Spark} utilizadas para substituir as funções no \textit{MapReduce}.   As funções listadas aqui foram obtidas da documentação original do \textit{Spark}\cite{SparkPage}, e apoiado nos exemplos do livro \cite{karau2015learning}.

Em face à primitiva \textit{Map} do \textit{MapReduce} será utilizada a primitiva 

\subsection{Funções do \textit{Spark} utilizadas no projeto}

Esta seção será dedicada a explicar alguns conceitos das operações realizadas pelo \textit{Spark}, realizando um paralelo entre suas operações e as operações do \textit{MapReduce}.

Segundo \cite{LearningSparkAl}, o \textit{Spark} utiliza \textit{RDDs} em seus nós de processamento, e permite que sejam realizadas \textbf{ações} e \textbf{transformações} como operação sobre os \textit{RDDs}.  \textbf{Ações} são operações que retornam um resultado para a aplicação ou armazenam  os dados e realizam um cálculo, \textit{count()} e \textit{first()} são exemplos de ações realizadas sobre os \textit{RDDs}. Já \textbf{transformações} realizadas sobre um determinado \textit{RDD} retorna um novo \textit{RDD}, por exemplo as operações de \textit{map()} e \textit{filter()}.

As operações de \textbf{ação} utilizadas no projeto são:
\begin{itemize}
\item \textit{\textbf{count}}: Operação que contabiliza a quantidade de elementos em um \textit{RDD}, geralmente a quantidade de linhas.ido em um \textit{RDD}.
\item \textit{\textbf{first}}: Retorna o primeiro elemento contido em um \textit{RDD}.
\item \textit{\textbf{reduce}}: Utiliza uma função que opera sobre dois elementos do mesmo tipo no \textit{RDD} e retorna um novo elemento do mesmo tipo. Por exemplo, para somar todos os elementos inteiros de um \textit{RDD}, podemos utilizar a função soma (+) dos inteiros, e obteremos como resultado um único valor inteiro correspondente à soma de todos elementos do \textit{RDD}. Pode ser utilizada em alguns casos como substituta da função de mesmo nome do \textit{MapReduce}, porém, a função mais adequada do \textit{Spark} em vários casos será a \textit{reduceByKey}.
\end{itemize}
As operações de \textbf{transformação} utilizadas no projeto são:
\begin{itemize}
\item \textit{\textbf{map}}: Esta operação tem como propósito aplicar uma determinada função em cada um dos elementos de um \textit{RDD} e retornar um novo \textit{RDD} contendo o resultado da função aplicada. Será utilizada ainda como substituta de função homônima no \textit{MapReduce}.
\item \textit{\textbf{flatMap}}: Semelhante ao \textit{map}, porém, cada item de entrada pode ser mapeado para zero ou mais itens de saída, portanto, a função deve retornar uma sequência de elementos em vez de um único item, geralmente utilizada para iterar nos elementos de uma lista que estão em uma única linha de um \textit{RDD} e retornar um elemento por linha do \textit{RDD}.
\item \textit{\textbf{mapPartition}}: Semelhante ao \textit{map}, mas é executado separadamente em cada partição (bloco) do RDD.
\item \textit{\textbf{reduceByKey}}: Utilizada quando os objetos contidos no \textit{RDD} são do tipo chave/valor, no qual os objetos que possuem mesma chave serão agrupados, e será aplicada a ação \textit{reduce}, utilizando uma função, sobre os valores que possuem chaves iguais. É elegível como substituta da função \textit{reduce} do \textit{MapReduce} em muitos casos.
\item \textit{\textbf{groupByKey}}: Possui funcionalidade próxima à função \textit{reduceByKey}, agrupando os valores baseados nas chaves em comum, porém, neste caso gera uma lista contendo os valores que possuem mesma chave agrupando por chave, não permitindo que sejam aplicadas funções sobre os valores. É elegível como substituta da função \textit{reduce} do \textit{MapReduce} dependendo da necessidade.
\item \textbf{\textit{aggregateByKey}}: Possui funcionalidade próxima à \textit{groupByKey}, porém quando chamado em um conjunto de dados de pares chave/valor, retorna um conjunto de dados de pares em que os valores de cada chave são agregados usando as funções combinadas fornecidas e um valor inicial neutro, nulo ou uma lista vazia. Permite tipos valores agregados que são diferente do tipo do valor contido nos pare, evitando alocações desnecessárias.
\item \textbf{\textit{filter}}: Retornar um novo conjunto de dados formado por elementos selecionados do \textit{RDD} origem em que passarem por uma função de verificação cujo retorno seja verdadeiro quando aplicada ao elemento.
\item \textbf{\textit{sortByKey}}: Aplicada aos \textit{RDDs} que contenham pares do tipo chave/valor, e retorna um novo \textit{RDD} que possua elementos ordenados pela chave do par, permitindo que seja selecionado se as chaves serão ordenadas em ordem crescente, ou decrescente.
\end{itemize}

As funções citadas acima são as mais utilizadas na implementação proposta do algorítimo, e foram expostas com a finalidade de facilitar a leitura, e entendimento do algorítimo implementado.










